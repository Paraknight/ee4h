After loading an image, the intermediary and final results of isolation and classification are stored in a data structure to adequately separate all the data for each real card as an equivalent \verb!C++! object. This is the Card class:

\begin{lstlisting}

class Card {
public:
\end{lstlisting}

\begin{itemize}
    \item Enumerations for the Suit, PictureRank and Colour attributes:
\end{itemize}

\begin{lstlisting}
	//Suit enumeration
	enum Suit {
	    CLUBS,
	    DIAMONDS,
	    HEARTS,
	    SPADES,
	    UNKNOWN_SUIT
	};

	//Picture card rank enumeration
	enum PictureRank {
	    RANK_JACK,
	    RANK_QUEEN,
	    RANK_KING,
	    RANK_ACE,
	    UNKNOWN_RANK
	};

	//Suit colour enumeration
	enum Colour {
	    BLACK,
	    RED,
	    UNKNOWN_COLOUR
	};
\end{lstlisting}

\begin{itemize}
    \item Configuration data for the final dimensions of the isolated card, the regions to be considered as the corners and GUI output colours:
\end{itemize}

\begin{lstlisting}
	static const int WIDTH = 250, HEIGHT = 350, AREA = 250*350, CORNER_AREA;
	static const cv::Rect TOP_CORNER_RECT;
	static const cv::Rect BOTTOM_CORNER_RECT;
	static const cv::Scalar LINE_COLOUR, LINE_COLOUR_ALT;
	static const cv::Scalar TEXT_COLOUR;
\end{lstlisting}

\begin{itemize}
	\item Data storage for intermediate \code{cv::Mat} objects used for classification as well as the object attributes that store the actual classification results themselves:
\end{itemize}

\begin{lstlisting}
	//Data
	cv::Mat mat, mat_clahe, mat_bin, mat_sym, mat_rank;
	Suit detected_suit;
	Colour detected_colour;
	PictureRank detected_rank;
	int detected_value;
	bool is_picture_card;
	cv::Rect _last_aabb, _rank_aabb;
\end{lstlisting}

\begin{itemize}
	\item Constructor and function prototypes for manipulation of the object within the larger scope of the program:
\end{itemize}

\begin{lstlisting}
	//Constructor declarations
	Card();
	Card(cv::Mat);

	//Functions
	void set_mat(cv::Mat mat);
	void init();
	void show();
	void show_with_card(cv::Mat card);
	cv::Mat results_to_mat();
	cv::Mat as_mat_with_results();
};
\end{lstlisting}

Using this object orientated approach to data storage simplifies the tasks of keeping track of isolated cv::Mats and classified attributes for each card in the input image, and also allows iteration through all cards in a \code{std::vector}. Most importantly it allows calling any of the classification functions such as \code{void detect_value_picture(Card *card)} by passing a pointer to the object and guaranteeing that the required member (for example \code{Card::mat_rank} for the isolated rank image matrix) is available for use within that function.

The usefulness of this capability is shown by the high level of abstraction possible in the main part of the software as shown below:

\begin{lstlisting}

for(size_t i = 0; i < cards.size(); i++)
{
    Card *card = &cards[i];

    //Detect suit colour
    detect_colour(card);

    //Detect suit type (number/picture)
    detect_type(card);

    //Find symbols
    find_symbols(card);

    //Get card value
    if (!card->is_picture_card)
    {
        detect_value_number(card);
    }
    else
    {
        detect_value_picture(card);
    }

    //Find suit
    find_suit_sym(card);
}

//Show results until key press
show_cascade(cards);
\end{lstlisting}